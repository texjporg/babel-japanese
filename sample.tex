%%
%% This is file `sample.tex',
%% generated with the docstrip utility.
%%
%% The original source files were:
%%
%% japanese.dtx  (with options: `sample')
%% 
%% File 'japanese.dtx'
%% Babel package for LaTeX version 2e
%% Copyright (C) 1989 - 2007
%%           by Johannes Braams, TeXniek
\documentclass{jbook}
\usepackage[german,english,japanese]{babel}
\makeatletter
\def\tbcaption{\def\@captype{table}\caption{キャプションの例}}
\def\fgcaption{\def\@captype{figure}\caption{キャプションの例}}
\makeatother
\def\yes{--- はい。}
\def\no{--- いいえ。}
\def\TEXT{Textverarbeitung mit einem Rechner kann in vielf\"altiger Weise
erfolgen. Eigenschaften und Leistungsf\"ahigkeit sind hierbei weniger
vom jeweiligen Rechnertype, sondern vielmehr vom verwendeten
\textit{Textverarbeitungsprogramm} bestimmt.}
\def\se{\selectlanguage{english}}
\def\sj{\selectlanguage{japanese}}
\def\sg{\selectlanguage{german}}
\setlength{\hoffset}{-13mm}
\setlength{\textwidth}{16cm}
\begin{document}

\chapter{babel}
\section{japaneseパッケージ}
japaneseパッケージは日本語による見出し語と日付を出力するためのマクロを
定義しています。

\fgcaption

\begin{itemize}
\se
\item ここで英語(\texttt{english})に変更します。(languageの値は\the\language)

\TEXT

ここは英語? \iflanguage{english}{\yes}{\no}\par
ここはドイツ語? \iflanguage{german}{\yes}{\no}\par
ここは日本語? \iflanguage{japanese}{\yes}{\no}

※ \verb:\adddialect\l@japanese0: と設定しているため,日本語?も「はい」となります。

\sg
\item ここでドイツ語(\texttt{german})に変更します。(languageの値は\the\language)

\TEXT

ここは英語? \iflanguage{english}{\yes}{\no}\par
ここはドイツ語? \iflanguage{german}{\yes}{\no}\par
ここは日本語? \iflanguage{japanese}{\yes}{\no}

※ ハイフネーションがドイツ語―旧正書法―に切り替わっている点に注目。
なお,新正書法(\texttt{ngerman})では\texttt{Leis-tungs-f\"a-hig-keit}のように分綴します。

\sj
\item ここで日本語(\texttt{japanese})に変更します。(languageの値は\the\language)
\tbcaption
\item \verb:\和暦: は日付の表示をデフォルトの西暦「\today 」から和暦「\和暦\today 」に変更します。
\end{itemize}
\end{document}
%%
%% \CharacterTable
%%  {Upper-case    \A\B\C\D\E\F\G\H\I\J\K\L\M\N\O\P\Q\R\S\T\U\V\W\X\Y\Z
%%   Lower-case    \a\b\c\d\e\f\g\h\i\j\k\l\m\n\o\p\q\r\s\t\u\v\w\x\y\z
%%   Digits        \0\1\2\3\4\5\6\7\8\9
%%   Exclamation   \!     Double quote  \"     Hash (number) \#
%%   Dollar        \$     Percent       \%     Ampersand     \&
%%   Acute accent  \'     Left paren    \(     Right paren   \)
%%   Asterisk      \*     Plus          \+     Comma         \,
%%   Minus         \-     Point         \.     Solidus       \/
%%   Colon         \:     Semicolon     \;     Less than     \<
%%   Equals        \=     Greater than  \>     Question mark \?
%%   Commercial at \@     Left bracket  \[     Backslash     \\
%%   Right bracket \]     Circumflex    \^     Underscore    \_
%%   Grave accent  \`     Left brace    \{     Vertical bar  \|
%%   Right brace   \}     Tilde         \~}
%%
\endinput
%%
%% End of file `sample.tex'.
